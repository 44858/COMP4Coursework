\chapter{Design}

\section{Overall System Design}

\subsection{Short description of the main parts of the system}
\begin{itemize}
	\item Stock Database System
	\begin{itemize}
		\item User Interface
		\item Adding a new product
		\item Deleting a product
		\item Searching for Stock
		\item Displaying information about Stock
	\end{itemize}
\end{itemize}

User Interface:
\begin{itemize}
	\item Upon launching the program, the user will be presented with an option of 4 buttons. The first button says "Add new Product", the second "Delete a product", the third "Search for a product" and the fourth "Exit"
	\item After selecting the "Add new Product" option, the user is taken to a new user interface where they can enter in information about a new product to the database.
	\item Clicking the "Delete a product" option will present the user with a new user interface where they are able to delete a product from the database.
	\item Clicking the "Search for a product" option wil present the user with a new user interface where they can search for a product from the database.
	\item Clicking the "Exit" option will close the program.
\end{itemize}
Adding a new product:
\begin{itemize}
	\item After clicking the "Add new Product" button on the main menu, the user is presented with a new user interface.
	\item On this new interface, the user is presented with a series of boxes with labels indicating the type of information that should be entered.
	\item When all the information has been entered, the user will be able to click the submit button at the bottom of the page, which will add the information as a new record in the database.
\end{itemize}

Deleting a product: 
\begin{itemize}
	\item After clicking the "Delete a product" button, the user will be taken to a new user interface.
	\item On this new interface
\subsection{System flowcharts showing an overview of the complete system}

\section{User Interface Designs}

\section{Hardware Specification}

\section{Program Structure}

\subsection{Top-down design structure charts}

\subsection{Algorithms in pseudo-code for each data transformation process}

\subsection{Object Diagrams}

\subsection{Class Definitions}

\section{Prototyping}

\section{Definition of Data Requirements}

\subsection{Identification of all data input items}

\subsection{Identification of all data output items}

\subsection{Explanation of how data output items are generated}

\subsection{Data Dictionary}

\subsection{Identification of appropriate storage media}

\section{Database Design}

\subsection{Normalisation}

\subsubsection{ER Diagrams}

\subsubsection{Entity Descriptions}

\subsubsection{1NF to 3NF}

\subsection{SQL Queries}

\section{Security and Integrity of the System and Data}

\subsection{Security and Integrity of Data}

\subsection{System Security}

\section{Validation}

\section{Testing}

\begin{landscape}
\subsection{Outline Plan}

\begin{center}
    \begin{tabular}{|p{2cm}|p{5cm}|p{5cm}|p{4cm}|}
        \hline
        \textbf{Test Series} & \textbf{Purpose of Test Series} & \textbf{Testing Strategy} & \textbf{Strategy Rationale}\\ \hline
        Example & Example & Example & Example \\ \hline
    \end{tabular}
\end{center}

\subsection{Detailed Plan}

\begin{center}
    \begin{longtable}{|p{1.5cm}|p{2.5cm}|p{2.5cm}|p{2cm}|p{2cm}|p{2cm}|p{2cm}|p{2cm}|}
        \hline
        \textbf{Test Series} & \textbf{Purpose of Test} & \textbf{Test Description} & \textbf{Test Data} & \textbf{Test Data Type (Normal/ Erroneous/ Boundary)} & \textbf{Expected Result} & \textbf{Actual Result} & \textbf{Evidence}\\ \hline
        Example & Example & Example & Example & Example & Example & Example & Example \\ \hline
    \end{longtable}
\end{center}
\end{landscape}